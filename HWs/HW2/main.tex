\documentclass{article}

\usepackage{graphicx}
\usepackage{fancyhdr}
\usepackage[sorting=none]{biblatex}
\usepackage[margin=1in]{geometry}
\usepackage{listings}
\usepackage[hidelinks]{hyperref}
\usepackage{subfigure}
\hypersetup{
    colorlinks=true,
    linkcolor=teal,
    filecolor=magenta,      
    urlcolor=teal,
    citecolor = teal
    }
\usepackage{xcolor}
\usepackage{xepersian}
\setlength\headheight{28pt} 
\addbibresource{bibliography.bib}
\settextfont[Path={./font/}, Scale=1.3]{IRLotus}
\setlatintextfont[Scale=1]{Times New Roman}
\usepackage{float} 
\renewcommand{\baselinestretch}{1.5}
\pagestyle{fancy}
\fancyhf{}
\rhead{\includegraphics[width=1cm]{img/Logo.png} پاسخ تمرین 2 }
\lhead{\thepage}
\renewcommand{\headrulewidth}{1pt}
\renewcommand{\footrulewidth}{1pt}
\AtBeginDocument{
	\def\chapterautorefname{فصل}%
	\def\sectionautorefname{پاسخ سوال}%
	\def\subsectionautorefname{بخش}%
	\def\subsubsectionautorefname{بخش}%
	\def\equationautorefname{رابطهٔ}%
    \def\lstlistingautorefname{برنامۀ}%
}
\renewcommand{\lstlistingname}{Code}

\definecolor{codegreen}{rgb}{0,0.6,0}
\usepackage{amsmath}
\definecolor{codegray}{rgb}{0.5,0.5,0.5}
\definecolor{codepurple}{rgb}{0.58,0,0.82}
\definecolor{backcolour}{rgb}{0.95,0.95,0.92}

\lstdefinestyle{mystyle}{
	backgroundcolor=\color{backcolour},   
	commentstyle=\color{codegreen},
	keywordstyle=\color{magenta},
	numberstyle=\tiny\color{codegray},
	stringstyle=\color{codepurple},
	basicstyle=\ttfamily\footnotesize,
	breakatwhitespace=false,         
	breaklines=true,                 
	captionpos=b,                    
	keepspaces=true,                 
	numbers=left,                    
	numbersep=5pt,                  
	showspaces=false,                
	showstringspaces=false,
	showtabs=false,                  
	tabsize=2
}

\lstset{style=mystyle}

\begin{document}

\begin{titlepage}
\begin{center}
\defpersianfont\nast[Path={./font/}, Scale=2]{IranNastaliq}
\centerline{{\includegraphics[width=5cm]{img/Logo.png}}}
\centerline{\textcolor[rgb]{0,0,0.5}{\nast \large  دانشگاه صنعتی خواجه نصیرالدین طوسی}}
\centerline{\textcolor[rgb]{0,0,0.5}{\nast \bfseries دانشکدۀ مهندسی برق - گروه مهندسی کنترل}}

\vfill
        
\Huge
\textbf{کنترل خطی}\\
\textbf{پاسخ تمرین 5}\\
        
\vfill
        
\begin{table}[ht]
    \centering
    \huge
    \begin{tabular}{|c|c|}
    \hline
    نام و نام خانوادگی & محمد امین حری فراهانی\\
    \hline
    شمارۀ دانشجویی & 40216674\\
    \hline
    تاریخ & آذر 1404\\
    \hline
    \end{tabular}
\end{table}
\end{center}
\end{titlepage}


\tableofcontents \clearpage
\listoffigures \clearpage
\clearpage
 \clearpage
\newpage



\section{قاعده میسون}\label{Section1}
قاعدهٔ میسون (\lr{Mason's Gain Formula}) روشی برای محاسبهٔ تابع انتقال در \lr{Signal Flow Graph} (نمودار جریان سیگنال) است. 
این قاعده با در نظر گرفتن تمام مسیرهای پیشرو، حلقه‌ها و حلقه‌های غیرهمبند، مقدار کلی بهرهٔ سیستم را بدون نیاز به حل معادلات جبری به‌دست می‌آورد.

فرمول کلی میسون به‌صورت زیر بیان می‌شود:

\begin{equation}
T = \frac{\sum_{k=1}^{N} P_k \Delta_k}{\Delta}
\end{equation}

که در آن:
\begin{itemize}
  \item $T$ : تابع انتقال کل سیستم
  \item $P_k$ : بهرهٔ مسیر پیشرو $k$ام
  \item $\Delta$ : تعیین‌کننده کلی سیستم، برابر است با
  \begin{latin}
  \[
  \Delta = 1 - (\text{Sum of individual loop gains}) + (\text{Sum of gain products of two non-touching loops}) - \cdots
  \]
  \end{latin}
  \item $\Delta_k$ : همان $\Delta$ است ولی با حذف حلقه‌هایی که با مسیر پیشرو $k$ام همبستگی دارند.
\end{itemize}


\LTRfootnote{SFG : Signal Flow Graph}
\subsection{رسم شماتیک SFG}
% --- اگر تصویر PDF داری ---
\begin{figure}[H]
  \centering
  \includegraphics[width=0.6\textwidth]{img/Asset_2.pdf}
  \caption{SFG سوال 1}
  \label{fig:block-diagram}
\end{figure}

\subsection{مراحل استفاده از فرمول میسون}

در ابتدا تعداد حلقه‌ها و بهره‌های مربوط به آن‌ها را می‌نویسیم:

\[
-G_{2}G_{3}H_{1}, \quad -G_{5}G_{6}H_{2}, \quad G_{7}G_{5}G_{8}, \quad -G_{7}G_{5}G_{9}G_{3}H_{1}
\]

از بین این چهار حلقهٔ موجود در نمودار، فقط دو حلقهٔ 
$-G_{2}G_{3}H_{1}$ 
و 
$-G_{5}G_{6}H_{2}$ 
دو به دو اشتراکی با یکدیگر ندارند. 

همچنین هیچ سه حلقه‌ای وجود ندارد که هیچ‌گونه اشتراکی با هم نداشته باشند.

در این بخش تابع تبدیل ورودی \(R_2\) به خروجی \(Y_1\) را با استفاده از قاعدهٔ میسون محاسبه می‌کنیم. برای این منظور ابتدا مسیرهای پیشرو و تعیین‌کننده‌ها (حلقه‌ها) را مشخص می‌کنیم.

فرض کنیم حلقه‌ها را به‌صورت زیر نام‌گذاری کنیم:
\[
\begin{aligned}
L_{1} &= -G_{2}G_{3}H_{1},\\
L_{2} &= -G_{5}G_{6}H_{2},\\
L_{3} &= G_{7}G_{5}G_{8},\\
L_{4} &= -G_{7}G_{5}G_{9}G_{3}H_{1},\\
\end{aligned}
\]

مسیرهای پیشرو از \(R_2\) به \(Y_1\) را به‌صورت :
\[
P_{1}=G_{11}G_{2}G_{3},\; P_{2}=G_{4}G_{5}G_{9}G_{3},\;P_{3}=G_{11}G_{7}G_{5}G_{9}G_{3},\;P_{4}=G_{4}G_{5}G_{8}G_{2}G_{3}
\]
است.

با توجه به مسئله: تنها زوج حلقه‌های غیرهم‌اشتراکی، \(L_{1}\) و \(L_{2}\) هستند و هیچ مجموعهٔ سه‌تایی غیرهم‌اشتراکی‌ای وجود ندارد. بنابراین تعیین‌کنندهٔ کلی \(\Delta\) برابر است با:
\[
\Delta = 1 - (L_{1}+L_{2}+L_{3}+L_{4}) + (L_{1}L_{2}),
\]

\[
=>\Delta = 1 - G_{7}G_{5}G_{8}+G_{2}G_{3}H_{1}+G_{5}G_{6}H_{2}+G_{7}G_{5}G_{9}G_{3}H_{2}+G_{2}G_{3}G_{5}G_{6}H_{1}H_{2}
\]
جملات حاصل‌ضرب سه‌تایی یا بیشتر صفرند.

برای هر مسیر پیشرو \(P_k\) نیز \(\Delta_k\) را به‌صورت \(\Delta\) اما با حذف حلقه‌هایی که با مسیر \(P_k\) اشتراک دارند تعریف می‌کنیم. 
(در عمل برای هر \(k\) باید حلقه‌هایی که با \(P_k\) تماسی ندارند برداشته و جمع/حاصل‌ضربِ متناظر محاسبه شود.)

در نهایت با قاعدهٔ میسون تابع تبدیل مورد نظر خواهد بود:
\[
\frac{Y_{1}}{R_{2}} \;=\; \frac{\displaystyle\sum_{k=1}^{4} P_{k}\,\Delta_{k}}{\Delta} = \frac{(G_{11}G_{2}G_{3})(-G_{5}G_{6}H_{2}) + (G_{4}G_{5}G_{9}G_{3})(1) + (G_{11}G_{7}G_{5}G_{9}G_{3})(1) + (G_{4}G_{5}G_{8}G_{2}G_{3})(1)}{1 - G_{7}G_{5}G_{8}+G_{2}G_{3}H_{1}+G_{5}G_{6}H_{2}+G_{7}G_{5}G_{9}G_{3}H_{2}+G_{2}G_{3}G_{5}G_{6}H_{1}H_{2}}
\]


حال مسیرهای پیشرو از \(E_7\) به \(E_4\) را مشخص میکنیم  :
\[
P_{1}=-G_{5}G_{9}G_{3}H_{2},\; P_{2}=-G_{5}G_{8}G_{2}G_{3}H_{2},\;
\]
 
لذا مانند قسمت الف) داریم :

\[
\frac{E_{4}}{E_{7}} \;=\; \frac{\displaystyle\sum_{k=1}^{2} P_{k}\,\Delta_{k}}{\Delta} = \frac{(-G_{5}G_{9}G_{3}H_{2})(1) + (-G_{5}G_{8}G_{2}G_{3}H_{2})(1) }{1 - G_{7}G_{5}G_{8}+G_{2}G_{3}H_{1}+G_{5}G_{6}H_{2}+G_{7}G_{5}G_{9}G_{3}H_{2}+G_{2}G_{3}G_{5}G_{6}H_{1}H_{2}}
\]

\section{رفع اغتشاش سوال 2}

\subsection{استفاده از فرمول میسون}

هدف کم‌کردن یا حذف اثر اغتشاش \(N(s)\) روی خروجی \(Y(s)\) با کمینه یا صفر کردن تابع انتقال \(T_{yn}(s)=\dfrac{Y(s)}{N(s)}\) است .
میتوانیم SFG شکل را رسم نموده و از طریق فرمول میسون  \(T_{yn}(s)=\dfrac{Y(s)}{N(s)}\)  را بدست اورده :
\begin{figure}[H]
  \centering
  \includegraphics[width=0.6\textwidth]{img/Asset_3.pdf}
  \caption{ SFG سیستم}
  \label{fig:block-diagram}
\end{figure}

برای اینکه نویز در خروجی حذف شود باید \(T_{yn}(s)=\dfrac{Y(s)}{N(s)}\)  کمینه یا صفر شود لذا نیازی به محاسبه  \(\Delta\) نیست و درواقع باید حاصل ضرب \(P_k\) \(\Delta_k\)  در فرمول میسون برابر صفر شود .

\indent % برای ایجاد تورفتگی در متن
مسیرهای پیشرو از \(Noise\) به \(Y\) به‌صورت زیر است :
\[
P_{1}=1\times1 ,\; P_{2}=-G \times \frac{s+10}{s+30} \times \frac{5}{s^2+10s}
\]

\[
P_{1} K_{1} + P_{2} K_{2} = 0 = 1 \times \left(1 + \frac{s + 10}{s + 30} \right) + (-G \times \frac{s+10}{s+30} \times \frac{5}{s^2+10s})(1)
\]

\[
\Rightarrow \quad G =\frac{2s}{5}(s+20)
\]


\pagebreak
\section{نمودار بلوکی}

\subsection{بخش اول}
الف ) با توجه به شکل 4 داریم :
\[
 [ -Y_{1}(\frac{s+1}{s^2+s+1}) + C ]G_{1} = Y_{1}
\]

\[
\Rightarrow \quad\frac{Y_{1}}{C} = \frac{G_{1}}{1 + G_{1}\left(\frac{s+1}{s^{2}+s+1}\right)} = \frac{s^2+s+1} 
{s^3+s+2}\Rightarrow \quad\ G_{1} = \frac{s^2+s+1}{s^3+1}\quad\ \]
که حالا میتوانیم به دلخواه تابع \(G_1\) را به دو تابع تبدیل جدا از هم تبدیل کنیم : 


\[
  G_{2} = \frac{1}{s+1}\quad , \quad  G_{3} = \frac{s^2+s+1}{s^2-s+1}
\]

\begin{figure}[H]
  \centering
  \includegraphics[width=0.6\textwidth]{img/Asset_4.pdf}
  \caption{ نمودار بلوکی}
  \label{fig:block-diagram}
\end{figure}
\subsection{بخش دوم}
ب ) با توجه به قسمت قبلی میتوانیم شکل نمودار را ساده کنیم پس داریم :

\begin{figure}[H]
  \centering
  \includegraphics[width=0.6\textwidth]{img/Asset_6.pdf}
  \caption{ نمودار بلوکی}
  \label{fig:block-diagram}
\end{figure}

\[
\Rightarrow \quad\ Y_{1} = (Y_{1} . C)(\frac{1}{s-1} )(\frac{2}{s})
\]

\[
\Rightarrow \quad\ \frac{Y_{1}}{C} = \frac{2}{s^2-s-2}
\]
\vspace*{2cm} 

\subsection{بخش سوم}
 ج) در این قسمت چون ورودی مستقیم به خروجی وصله یعنی سیستم مستقل از عناصره :

\begin{figure}[H]
  \centering
  \includegraphics[width=0.6\textwidth]{img/Asset_7.pdf}
  \caption{ نمودار بلوکی}
  \label{fig:block-diagram}
\end{figure}

\[
\Rightarrow \quad\ \frac{V_{out}}{V_{in}} = 1
\]

\section{بررسی موتور  DC}


\begin{figure}[H]
  \centering
  \includegraphics[width=0.6\textwidth]{img/dc.png}
  \caption{ ماشین DC}
  \label{fig:block-diagram}
\end{figure}
\subsection{معادلات حاکم بر سیستم}

در ابتدا پارامتر هایی که در فایل متلب بود رو مورد بررسی قرار میدهیم :
\[
\begin{aligned}
&J = 0.01 &&\text{(ممان اینرسی)}\\
&b = 0.1  &&\text{(ضریب اصطکاک )}\\
&K = 0.01 &&\text{(ثابت طراحی)}\\
&R = 1    &&\text{(مقاومت آرمیچر)}\\
&L = 0.5  &&\text{(اندوکتانس )}
\end{aligned}
\]
اکنون معادلات مغناطیسی و الکتریکی حاکم بر سیستم را مینویسیم :

\[
\begin{cases}
J\,\dfrac{d\omega(t)}{dt} + b\,\omega(t) = K\,i(t),\\[6pt]
L\,\dfrac{di(t)}{dt} + R\,i(t) + K\,\omega(t) = V(t).
\end{cases}
\]
که با توجه به مقادیر داخل فایل : 
\[
\begin{cases}
0.01\;\dfrac{d\omega(t)}{dt} + 0.1\,\omega(t) = 0.01\,i(t),\\[6pt]
0.5\;\dfrac{di(t)}{dt} + 1\,i(t) + 0.01\,\omega(t) = V(t).
\end{cases}
\]
حال معادلات را در حوزه لاپلاس میبریم و تابع تبدیل را بدست میاوریم :
\[
\begin{aligned}
&J s\,\Omega(s) - J\,\omega(0) + b\,\Omega(s) = K\,I(s),\\[6pt]
&L s\,I(s) - L\,i(0) + R\,I(s) + K\,\Omega(s) = V(s).
\end{aligned}
\]

% --- اگر شرایط اولیه صفر باشند (\omega(0)=0,\; i(0)=0) ---
\[
\begin{aligned}
&(J s + b)\,\Omega(s) = K\,I(s),\\[6pt]
&(L s + R)\,I(s) + K\,\Omega(s) = V(s).
\end{aligned}
\]

% --- حل جبری برای بدست آوردن تابع تبدیل \Omega(s)/V(s) (نمادین) ---
\[
\frac{\Omega(s)}{V(s)}
= \frac{K}{LJ\,s^{2} + (Lb + RJ)\,s + \big(Rb + K^{2}\big)}.
\]

% --- جایگذاری مقادیر عددی (J=0.01,\; b=0.1,\; K=0.01,\; R=1,\; L=0.5) ---
\[
\frac{\Omega(s)}{V(s)}
= \frac{0.01}{0.005\,s^{2} + 0.06\,s + 0.1001} = \frac{100}{50\,s^{2} + 600\,s + 1001}  .
\]
% معادلات به صورت حالت (state-space) — مفید برای پیاده‌سازی در Simulink
با توجه به تابع تبدیل میتوان مقادیر فرکانس طبیعی نامیرا و نسبت میرایی را حساب کرد :

\[
\Rightarrow \omega_{n} = \sqrt{\frac{1001}{50}} \approx 4.474 , \quad 2\zeta\omega_{n} = 12 , \Rightarrow \zeta \approx 1.34
\]
\begin{RTL}
چون $\zeta > 1$ شد، پس می‌توان فهمید که قطب‌ها حقیقی هستند و پاسخ پلهٔ این سیستم در حالت کلی به صورت زیر است:
\end{RTL}

\[
y(t) = K(1 + \frac{\omega_{n}}{2\sqrt{\zeta^2-1}}(\frac{e^{-s_{1}t}}{s_{1}} - \frac{e^{-s_{2}t}}{s_{2}}))
\]
مطابق خروجی متلب در ورودی پله  :
\begin{figure}[H]
  \centering
  \includegraphics[width=0.6\textwidth]{img/step.png}
  \caption{  Response Step}
  \label{fig:block-diagram}
\end{figure}

\subsection{طراحی مدل ماشین DC در سیمولینک متلب}

با توجه به معادلات سیستم میتوان چنین سیستمی را برای ماشین DC مدل کرد :
\begin{figure}[H]
  \centering
  \includegraphics[width=1.1\textwidth]{img/myModel.pdf}
  \caption{Model Machine DC}
  \label{fig:block-diagram}
\end{figure}

که نیمه اول مدل مربوط به رابطه الکتریکی حاکم بر ماشین DC می شود و بقیه آن مربوط به رابطه مکانیکی و همچنین مشاهده میشود از بلوک های بهره و انتگرال گیر در طراحی استفاده شده است.



\begin{figure}[h!]
    \centering
    \begin{minipage}{0.48\textwidth}
        \centering
        \includegraphics[width=\textwidth]{img/speed_my.png}
        \caption{ورودی پله به سیستم مدل شده}
        \label{fig:image1}
    \end{minipage}
    \hfill
    \begin{minipage}{0.48\textwidth}
        \centering
        \includegraphics[width=\textwidth]{img/step.png}
        \caption{فایل متلب داده شده}
        \label{fig:image2}
    \end{minipage}
\end{figure}
\vspace{2cm}
\begin{RTL}
برای فشرده‌سازی مدل می‌توان بلوک‌ها را به صورت یک \lr{Subsystem} ایجاد کرد و تنها ورودی‌ها و خروجی‌های اصلی را به آن متصل نمود.
در این سیستم، ورودی‌ها شامل \(V(t)\) و یا \(T_{l}\) هستند و خروجی‌ها نیز \(\omega\) و \(i\) می‌باشند.
همچنین می‌توان بخش‌های الکتریکی یا مکانیکی موتور را به صورت جداگانه در قالب \lr{Subsystem} قرار داد تا مدار کلی فشرده‌تر و منظم‌تر شود.
\end{RTL}




\section{ضمیمه}
\href{https://github.com/MJAHMADEE/ARASLaTeXFormats}{ ادرس گیت هاب  (\lr{GitHub})}.

%\printbibliography[title=منابع]

\begin{thebibliography}{99}


\end{thebibliography}

\renewcommand{\section}[2]{}%
\begin{thebibliography}{99} % assumes less than 100 references
%چنانچه مرجع فارسی نیز داشته باشید باید دستور فوق را فعال کنید و مراجع فارسی خود را بعد از این دستور وارد کنید
\begin{LTRitems}
\resetlatinfont
\bibitem{mason1956}
S.~J. Mason, ``Feedback Theory—Further Properties of Signal Flow Graphs,'' 
\emph{Proceedings of the I.R.E.}, vol.~44, no.~7, July 1956.
% PDF (MIT RLE tech. report reprint): https://dspace.mit.edu/bitstream/handle/1721.1/4778/RLE-TR-303-15342712.pdf

\bibitem{ogata2010}
K.~Ogata, \emph{Modern Control Engineering}, 5th ed. Prentice Hall, 2010.

\bibitem{nise2014}
N.~S. Nise, \emph{Control Systems Engineering}, 7th ed. Wiley, 2014.

\bibitem{mathworks_dc}
The MathWorks, ``DC Motor Control — MATLAB \& Simulink Example,'' 
\emph{MathWorks Documentation}. Available: \url{https://www.mathworks.com/help/control/ug/dc-motor-control.html}

\bibitem{kuczmann2024}
M.~Kuczmann, ``Review of DC Motor Modeling and Linear Control Theory with Laboratory Tests,'' 2024. 
(Review article — modeling, identification and control of DC motors.) 
Available (preprint / repository): \url{https://www.researchgate.net/publication/381246468_Review_of_DC_Motor_Modeling_and_Linear_Control_Theory_with_Laboratory_Tests}

\bibitem{ctms}
R.~K. Kraus (or CTMS authors), ``DC Motor Speed: System Modeling,'' \emph{Control Tutorials for MATLAB and Simulink (CTMS)}, University of Michigan. 
Available: \url{https://ctms.engin.umich.edu/CTMS/index.php?example=MotorSpeed&section=SystemModeling}

\end{LTRitems}

\end{thebibliography}

\end{document}
