\documentclass{article}
\usepackage{amsmath}

\usepackage[final]{neurips}
\usepackage{subfigure}
\usepackage{enumitem}
\usepackage{subfloat}
% to avoid loading the natbib package, add option nonatbib:
    % \usepackage[nonatbib]{neurips_2019}
\usepackage{setspace} % Include the setspace package in your preamble
\usepackage{listings}
\usepackage[dvipsnames]{xcolor}
% To set one-and-a-half spacing:
\onehalfspacing
\usepackage[utf8]{inputenc} % allow utf-8 input
% \usepackage{natbib}
% \let\citep\cite
% \let\citelatin\cite
% \let\citeplatin\cite

\usepackage[T1]{fontenc}    % use 8-bit T1 fonts
\usepackage[hidelinks]{hyperref}      % hyperlinks
\hypersetup{
    colorlinks=true,
    linkcolor=red,
    filecolor=magenta,      
    urlcolor=blue,
    citecolor = teal
    }
% \usepackage{xcolor}
\usepackage{tcolorbox}
\setcounter{footnote}{0} 
\usepackage{amsmath} 
\usepackage{url}            % simple URL typesetting
\usepackage{booktabs}       % professional-quality tables
\usepackage{amsfonts}       % blackboard math symbols
\usepackage{nicefrac}       % compact symbols for 1/2, etc.
\usepackage{microtype}      % microtypography
\usepackage{graphicx}
\usepackage{float}
\usepackage{xepersian}
\usepackage{fontawesome5}

\settextfont{XB Yas.ttf}


\title{محمدامین حری فراهانی- پاسخ تمرین 4}





\renewcommand{\lstlistingname}{Code}

\definecolor{codegreen}{rgb}{0,0.6,0}
\definecolor{codegray}{rgb}{0.5,0.5,0.5}
\definecolor{codepurple}{rgb}{0.58,0,0.82}
\definecolor{backcolour}{rgb}{0.95,0.95,0.92}

\lstdefinestyle{mystyle}{
	backgroundcolor=\color{backcolour},   
	commentstyle=\color{codegreen},
	keywordstyle=\color{magenta},
	numberstyle=\tiny\color{codegray},
	stringstyle=\color{codepurple},
	basicstyle=\ttfamily\footnotesize,
	breakatwhitespace=false,         
	breaklines=true,                 
	captionpos=b,                    
	keepspaces=true,                 
	numbers=left,                    
	numbersep=5pt,                  
	showspaces=false,                
	showstringspaces=false,
	showtabs=false,                  
	tabsize=2
}

\lstset{style=mystyle}

\begin{document}


\begin{minipage}{0.1\textwidth}% adapt widths of minipages to your needs
\includegraphics[width=1.3cm]{img/Logo.png}
\end{minipage}%
\hfill%
\begin{minipage}{0.9\textwidth}\raggedleft
دانشگاه صنعتی خواجه نصیرالدین طوسی تهران\\
پاسخ تمرین چهارم\\
\end{minipage}
% \end{}

\makepertitle

\begin{center}
\href{https://github.com/aminism00/LinearControl2025/tree/main/HWs/HW4}{\faGithub\ مخزن گیت هاب}
\end{center}


\section{سوال اول}
\subsection{(الف)}

تابع انتقال مورد نظر برابر است با:
\[
H(s) = (1+k_1 s)\left(\frac{3}{s^4+3s^3+5s^2+6s+k_2}\right)
\]

\[
\Delta(s)=s^4+3s^3+5s^2+6s+k_2
\]

جدول راث برای $\Delta(s)$:
\[
\begin{array}{c|ccc}
\text{ردیف} & \text{ستون 1} & \text{ستون 2} & \text{ستون 3} \\ \hline
s^4 & 1     & 5   & k_2 \\
s^3 & 3     & 6   & 0   \\
s^2 & \dfrac{3\cdot 5 - 1\cdot 6}{3} = 3 & \dfrac{3\cdot k_2 - 1\cdot 0}{3} = k_2 & 0 \\
s^1 & \dfrac{3\cdot 6 - 3\cdot k_2}{3} = 6-k_2 & 0 & 0 \\
s^0 & k_2 &  & 
\end{array}
\]

 میدانیم که برای پایداری همهٔ اعضای ستون اول باید هم علامت باشند که اینجا به معنی مثبت بودن است:
\[
\begin{aligned}
&1 > 0,\\
&3 > 0,\\
&3 > 0,\\
&6 - k_2 > 0 \quad\Rightarrow\quad k_2 < 6,\\
&k_2 > 0.
\end{aligned}
\]
بنابراین شرط نهایی برای پایداری:
\[
\boxed{0 < k_2 < 6}
\]

ضریب $k_1$ در تأثیری روی جای قطب‌ها ندارد، پس شرط پایداری فقط وابسته به $k_2$ است.
\subsection{(ب)}
تابع انتقال:
\[
H(s)=\frac{(1+k_1 s)G(s)}{1+(1+k_1 s)G(s)}
\]

\[
\Delta(s)=s^4+3s^3+5s^2+6s+k_2 + 3 + 3k_1 s
\]
برابر است با:
\[
\Delta(s)=s^4 + 3s^3 + 5s^2 + (6+3k_1)s + (k_2+3)
\]

\bigskip
\noindent جدول راث (بررسی پایداری):
\[
\begin{array}{c|ccc}
\text{ردیف} & \text{ستون 1} & \text{ستون 2} & \text{ستون 3} \\ \hline
s^4 & 1 & 5 & k_2+3 \\
s^3 & 3 & 6+3k_1 & 0 \\[6pt]
s^2 & \dfrac{3\cdot 5 - 1(6+3k_1)}{3}
= \dfrac{15-6-3k_1}{3} = 3 - k_1
& \dfrac{3(k_2+3)-1\cdot 0}{3} = k_2+3
& 0 \\[10pt]
s^1 & \displaystyle
\frac{(3-k_1)(6+3k_1) - 3(k_2+3)}{3-k_1}
= \frac{3\big(3 + k_1 - k_1^2 - k_2\big)}{3-k_1}
& 0 & 0 \\[12pt]
s^0 & k_2+3 & & 
\end{array}
\]
برای پایداری باید همهٔ اعضای ستون اول مثبت باشند:
\[
\begin{aligned}
&1 > 0,\\
&3 > 0,\\
&3 - k_1 > 0 \quad\Rightarrow\quad k_1 < 3,\\
&\frac{3\big(3 + k_1 - k_1^2 - k_2\big)}{3-k_1} > 0,\\
&k_2 + 3 > 0 \quad\Rightarrow\quad k_2 > -3.
\end{aligned}
\]

از انجایی که \(3-k_1>0\) , کافی است شرط مثبت بودن صورت کسر را بررسی کنیم
\[
3 + k_1 - k_1^2 - k_2 > 0 \quad\Rightarrow\quad k_2 < 3 + k_1 - k_1^2.
\]

در نتیجه شرط‌های نهایی پایداری عبارت‌اند از:
\[
\boxed{\;k_1 < 3,\qquad -3 < k_2 < 3 + k_1 - k_1^2\;}
\]
\subsection{(ج)}
برای اینکه هم سیستم حلقه بسته و هم سیستم حلقه ی باز پایدار باشد باید دو شرط هم زمان برقرار باشد
پس داریم:


\[
\boxed{0 < k_2 < 6 \qquad\;k_1 < 3,\qquad -3 < k_2 < 3 + k_1 - k_1^2\;}
\]



\section{سوال دو}
ابتدا چند جمله ای مخرج را مشخص میکنیم و بعد ان را گسترش می دهیم:
\[
\Delta(s)=(s^2-a^2)(s^2+b^2)(s+c).
\]

\[
\begin{aligned}
(s^2-a^2)(s^2+b^2) &= s^4 + (b^2-a^2)s^2 - a^2 b^2,\\
\Delta(s)&=\big(s^4 + (b^2-a^2)s^2 - a^2 b^2\big)(s+c)\\
&= s^5 + c s^4 + (b^2-a^2)s^3 + c(b^2-a^2)s^2 - a^2 b^2 s - c a^2 b^2.
\end{aligned}
\begin{array}{c|ccc}
\text{ردیف} & \text{ستون 1} & \text{ستون 2} & \text{ستون 3} \\ \hline
s^5 & 1 & b^2 - a^2 & -a^2 b^2 \\[6pt]
s^4 & c & c(b^2 - a^2) & -c a^2 b^2 \\[10pt]
s^3 & 4c & 2c(b^2 - a^2) & 0 \\[10pt]
s^2 &
\displaystyle \frac{4c \cdot c(b^2-a^2) - c\cdot 2c(b^2-a^2)}{4c}
= \frac{c(b^2-a^2)}{2}
&
- c a^2 b^2
& 0 \\[14pt]
s^1 &
\displaystyle
\frac{ \frac{c(b^2-a^2)}{2}\cdot 2c(b^2-a^2) - 4c(-c a^2 b^2)}
{ \frac{c(b^2-a^2)}{2} }
=
\frac{2c(a^2+b^2)^2}{\,b^2-a^2\,}
& 0 & 0 \\[14pt]
s^0 & - c a^2 b^2 & & \\
\end{array}
\]
سطر $s^3$ در محاسبهٔ مستقیم مقدار صفر به‌وجود آمد؛
بنابراین از چندجمله‌ای زیر استفاده می‌کنیم و مشتق ان را حساب می کنیم:
\[
A(s)=c\big(s^{4}+(b^{2}-a^{2})s^{2}-a^{2}b^{2}\big)
\]

\[
A'(s)=c\big(4s^{3}+2(b^{2}-a^{2})s\big)
\]
\\
میدانیم : 
$a,b,c>0 ,
p>0$
با این نکته و  جدول راث علامت ها را به راحتی تعیین میکنیم:

\[
\begin{aligned}
s^5 &: \;1 =x \quad >0,\\
s^4 &: \;c=y>0,\\
s^3 &: \;4c=z>0,\\
s^2 &: \;\frac{c(b^2-a^2)}{2} =p>0,\\
s^1 &: \;\frac{2c\,(a^2+b^2)^2}{\,b^2-a^2\,}=q>  0,\\
s^0 &: \;-c a^2 b^2=r<0.
\end{aligned}
\]

\section{سوال سه}
ابتدا معادله مشخصه ها ساده می کنیم و انها را به فرم 
$1+KG(s)$
در می آوریم.

\[
G(s)=\frac{s+3}{s(s+5)(s+6)(s^{2}+2s+2)}.
\]

 نقاط شکست:

\[
\frac{dG(s)}{ds}=0
\quad\Rightarrow\quad
s \approx \{-5.52,\,-0.65 \pm 0.46j,\,-3.33 \pm 1.20j\}.
\]

 مجانب‌ها:

\[
\sigma=\frac{\sum p_i - \sum z_i}{n-m}
=
\frac{(0-5-6-1-1)-(-3)}{5-1}
=
\frac{-13+3}{4}
=
-2.5.
\]

زاویه‌های مجانب برای:
\\
\(:K>0\)


\[
\theta=\frac{(2k+1)\pi}{n-m}
=\frac{(2k+1)\pi}{4}
\qquad
\theta=\frac{\pi}{4},\,\frac{3\pi}{4},\,\frac{5\pi}{4},\,\frac{7\pi}{4}
\]

 \(K<0\):

\[
\theta=\frac{2k\pi}{n-m}
=\frac{2k\pi}{4},
\qquad
\theta=0,\,\frac{\pi}{2},\,\pi,\,\frac{3\pi}{2}
\]

زاویهٔ خروج از قطب مختلط:

\[
\tan^{-1}\left(\tfrac{1}{2}\right)
-
\left(
135^\circ
+ \tan^{-1}\tfrac{1}{4}
+ \tan^{-1}\tfrac{1}{5}
+ \theta
\right)
=(2k+1)\pi.
\]

\[
\Theta \approx -43^\circ.
\]

فرکانس نوسانات دائمی:

\[
1+kG(s)=0
\quad\Rightarrow\quad
s^{5}+13s^{4}+54s^{3}+82s^{2}+(60+k)s+3k=0.
\]





\[
\begin{array}{c|ccc}
s^{5} & 1 & 54 & 60+k \\[4pt]
s^{4} & 13 & 82 & 3k \\[4pt]
s^{3} & \tfrac{620}{13} & 60+\tfrac{10}{13}k & 0 \\[6pt]
s^{2} & \tfrac{2035}{31}-\tfrac{13}{62}k & 3k & 0 \\[6pt]
s^{1} & \tfrac{10(k^{2}+652k-24420)}{13k-4070} & 0 & 0 \\[6pt]
s^{0} & 3k & &
\end{array}
\]

برای عبور ریشه‌ها از محور موهومی، سطر \(s^{1}\) باید صفر شود:
\[
10(k^{2}+652k-24420)=0.
\]


\[
k^2-625k-2440= 0 \quad\Rightarrow\quad k=36.89 , -661.89\times
\]
\[
(\frac{2035}{31}-\frac{13}{62}k)s^2 + 3k = 0\qquad \Rightarrow s=\pm1.38j
\]
بنابراین فرکانس نوسانات برابر است با:
\[
{\omega=1.38}
\]
\begin{figure}[h]
	\centering
    \includegraphics[scale=0.25]{3-1.jpg}
\end{figure}

\[
G(s)=\frac{s+1}{s^{2}(s+1)(5+s)}
\]

  نقاط شکست:

\[
\frac{dG(s)}{ds}
=\frac{-15s^{3}+69s^{2}+43s+6}{s^{3}(s+1)^{2}(s+5)^{2}}
\]

\[
s=\{-3.8\pm 1.2j,\,-3.5\}
\]
مجانب ها:
\[
\sigma=\frac{\sum p_i - \sum z_i}{n-m}
=\frac{(-1-5)-( -1)}{4-1}
=\frac{-29}{15}.
\]
زاویه های مجانب برای:
\[
\theta = \frac{(2k+1)\pi}{n-m}
=\frac{(2k+1)\pi}{3} \qquad
\theta=\frac{\pi}{3},\pi,\frac{5\pi}{3} \qquad
K>0
\]
\[
\theta = \frac{(2k)\pi}{n-m}
=\frac{(2k)\pi}{3} \qquad
\theta=0,\frac{2\pi}{3},\frac{4\pi}{3} \qquad
K<0
\]

معادلهٔ مشخصه برابر است با:
\[
1+kG(s)=0
\quad\Rightarrow\quad
s^{4}+6s^{3}+5s^{2}+5ks+k=0
\]

جدول راث را برای بررسی پایداری میسازیم:
\[
\begin{array}{c|ccc}
s^{4} & 1 & 5 & k \\[4pt]
s^{3} & 6 & 5k & 0 \\[4pt]
s^{2} & 5-\tfrac{5}{6}k & k & 0 \\[4pt]
s^{1} & \tfrac{5k-6k}{5-\frac{5}{6}k} & 0 & 0 \\[6pt]
s^{0} & k & &
\end{array}
\]

برای عبور ریشه‌ها از محور موهومی، سطر \(s^{1}\) باید صفر شود:
\[
5k - 6k = 0
\quad\Rightarrow\quad
k=\frac{6\times 19}{25}
\]

\[
s^{2}\left(5-\frac{5}{6}k\right)+k=0
\]

جایگذاری مقدار \(k\):
\[
s^{2}=\frac{19}{5}
\quad\Rightarrow\quad
s=\pm j\sqrt{\frac{19}{5}}
\]

بنابراین فرکانس نوسان برابر است با:
\[
\omega=\sqrt{\frac{19}{5}}.
\]

\begin{figure}[h]
	\centering
	\includegraphics[scale=0.25]{3-2.jpg}
\end{figure}


\section{پرسش چهارم}

\subsection{(الف)}
سیستم ما دارای دو قطب در 
$0 , -z $
و دو صفر در 
$2 ,-p $
است.
\\
همچنین می دانیم که
برای پایدار بودن نیاز است تا قطبی در سمت راست وجود نداشته باشد .
در مکان هندسی ریشه ها نیز قطب ها به سمت صفر ها حرکت میکنند .
در نتیجه میتوان برای پایدار سازی ,دو قطب را در سمت راست گذاشت در این صورت به ازای مقادیری از 
$k$
تمام قطب ها در سمت  چپ قرار میگیرند و سیستم پایدار میشود.
\\
 برای مثال یک حالت میتواند این باشد که داریم:
$p=-2 , z=2$

\[
L(s)=G_c(s)G_p(s)
\]

\[
L(s)=\frac{(s-2)(s+p)}{s(s+2)}
\]

\[
p=-2 \quad , \quad z=2
\]
داریم:
\[
L(s)=\frac{(s-2)^2}{s(s+2)}
\]

\[
\Delta(s)=1+kL(s)
\]

\[
\Delta(s)=s^2+2s+5+k(s^2-4s+4)
\]

\[
\Delta(s)=(k+1)s^2+(2-4k)s+4k
\]

شرایط پایداری:
\[
\begin{cases}
k>-1 \\
k<\frac{1}{2} \\
k>0
\end{cases}
\quad \Rightarrow \quad
0<k<\frac{1}{2}
\]

\begin{figure}[h]
	\centering
\includegraphics[scale=0.35]{4-1.png}
\end{figure}

\newpage
\subsection{(ب)}
در این قسمت باید نسبت میرایی بین  
$0.4 , 0.6$
باشد.
\[
\cos \theta = \zeta
\]
در نتیجه، تمام نقاطی که $\zeta=0.4$ دارند روی خطی از مبدأ قرار می‌گیرند
که با محور منفی $x$ زاویهٔ $76^\circ$ می‌سازد.\\
همچنین تمام نقاطی که $\zeta=0.6$ دارند روی خطی قرار می‌گیرند
که با محور منفی $x$ زاویهٔ $53^\circ$ می‌سازد.

در نتیجه تمام نقاط بین این خط که روی مکان هندسی ریشه ها قرار دارند قابل قبول است.
\begin{figure}[h]
	\centering
\includegraphics[scale=0.35]{4-2.png}
\end{figure}
\\
حال باید k محل برخورد دو خط را با مکان هندسی ریشه ها پیدا کنیم.

هر نقطه روی خط
$\zeta = 0.4 $
به فرم مقابل است:
\[
\tan\theta=\tan(\arccos 0.4) = 2.29  \qquad s = -\beta +2.29\beta j
\]

\[
1 + k\,G(s) = 0
\]

\[
G(s)=\frac{(s-2)^2}{s(s+2)}
\]

\[
\Rightarrow \Delta=(k+1)s^{2} + (2-4k)s + 4k = 0
\]
باید s را درون معادله جایگذاری کنیم تا مقدار 
$k ,\beta$
بدست بیاید.

	
	\[
	(k+1)(\beta+0.4\beta j)^2+(2-4k)(\beta+0.4\beta j)+4k=0
	\]
                     
 قسمت حقیقی و موهومی را جدا می‌کنیم و مقدار 
 $k , \beta$
	را بدست میاوریم.
	
	\begin{align*}
	\beta_1 &=0, 
	& k_1 &= 0,\\
	\beta_2 &=- 0.36, 
	& k_2 &=0.26,\\
	\beta_3 &=0.58, & k_3 &=1.11.
	\end{align*}
	\\
	واضح است که k مد نظر ما برابر با
	$0.26$
	و مقادیر دیگر k یا از مبدا گذر میکند و یا در ربع چهارم صفحه مختصات قرار میگرد که مورد تایید نیست چرا که سیستم ناپایدار میشود.
\\
فرکانس طبیعی مینیمم در نقاط برخورد رخ میدهد و بخش حقیقی نقاط برخورد همان
$\beta$
است.
\\
$-\zeta\omega_{n}= \beta \Rightarrow \qquad \omega_{n} = \frac{-0.36}{-0.4}=0.9$
\\
همین مراحل را برای زتای
$0.6$
تکرار میکنیم.
\[
\tan\theta=\tan(\arccos 0.6) = \frac{4}{3}  \qquad s = --\beta +\frac{4}{3}\beta j
\]
\begin{align*}
\beta_1 &=0, 
& k_1 &= 0,\\
\beta_2 &=0.49, 
& k_2 &=0.20,\\
\beta_3 &=-0.97, & k_3 &=1.92.
\end{align*}

باز هم مشابه ی قبل تنها 
$k_{3}$
مورد تایید است.
\[
-\zeta\omega_{n}= \beta \Rightarrow \qquad \omega_{n} = \frac{-0.97}{-0.6}=1.61
\]
در نتیجه فرکانس طبیعی مینیمم برای حالت اول است و مقدار آن برابر با 
$0.9$

\section*{سوال پنجم}
	
\begin{figure}[H]
	\centering
	\includegraphics[width=0.88\textwidth]{pic/6.jpg} 
	\label{fig2}
\end{figure}


\begin{figure}[H]
	\centering
	\includegraphics[width=0.88\textwidth]{pic/7.jpg} 
	\label{fig2}
\end{figure}

\section{نقشهٔ منطقهٔ پایدار در صفحهٔ \(\mathbf{k_1}\)--\(\mathbf{k_2}\)}
توضیح: این اسکریپت ناحیه‌ای را نمایش می‌دهد که با محدودیت‌های \(k_2>0\)، \(k_2<-k_1^2+k_1+3\)، \(k_2<6\) و \(k_1<3\) مطابقت دارد. همچنین مرزها و خطوط مرجع را می‌کشد.



است

\[
G(s)=\frac{1}{s(s+3)(s+6)}
\]
نقاط شکست:
\[
\frac{dG(s)}{ds}
=\frac{3s^{2}+18s+18}{s^{2}(s+3)^{2}(s+6)^{2}}=0
\quad\Rightarrow\quad
s=-1.26,\,-4.73
\]
مجانب ها:
\[
\sigma=\frac{\sum p_i - \sum z_i}{n-m}
=\frac{-3-6}{3}=-3
\]
زاویه خروج:
\[
\theta=\frac{(2k+1)\pi}{\,n-m\,}
=\left\{\frac{\pi}{3},\;\pi,\;\frac{5\pi}{3}\right\}
\]

\[
\Delta = 1 + kG(s)
\]

\[
\Delta = s^{3} + 9s^{2} + 18s + k
\]

جدول راث برای بررسی پایداری:

\[
\begin{array}{c|cc}
s^{3} & 1 & 18 \\
s^{2} & 9 & k \\
s^{1} & \frac{18 - k/9}{} & 0 \\
s^{0} & k &
\end{array}
\]

\[
k = 18\times 9 = 162
\]

\[
0<k<162
\]

\begin{figure}[h]
	\centering
	\includegraphics[scale=0.35]{5-1.png}
\end{figure}
\subsection{ب.}
\[
M_p < 20\% 
\;\Rightarrow\;
e^{-\frac{\pi \zeta}{\sqrt{1-\zeta^{2}}}} < 0.2
\;\Rightarrow\;
\zeta > 0.45
\;\Rightarrow\;
\cos\theta > 0.45
\;\Rightarrow\;
-63^\circ < \theta < 63^\circ
\]

\[
t_s < 2 
\;\Rightarrow\;
3\omega_n > \frac{-\ln\!\left[0.02\sqrt{1-\zeta^{2}}\right]}{2}
\;\Rightarrow\;
\zeta\omega_n > 2.01
\]
با توجه به دو رابطه بدست آمده مکان هندسی ریشه ها باید در ناحیه کشیده شده باشد.

\begin{figure}[h]
	\centering
	\includegraphics[scale=0.35]{5-2.png}
\end{figure}
مشاهده میکنیم که به ازای هیچ مقداری از 
$k$
,
3 ریشه در ناحیه ی مد نظر ما قرار نمیگیرد.
\subsection{ج.}
 یک کنترل کننده  pd اضافه میکنیم.
\\
یک صفر به ناحیه سمت چپ اضافه میکنیم به طوریکه قطب ها را به ناحیه مطلوب مان بکشد.
\\
برای مثال یک صفر در منفی$ 2.5 $اضافه میکنیم و بار دیگر مکان هندسی ریشه ها را رسم میکنیم.
\begin{figure}[h]
	\centering
	\includegraphics[scale=0.35]{5-3.png}
\end{figure}
\\
این بار مشاهده میکنیم که نقاطی وجود دارند که هر دو شرط بدست آمده را فراهم میکنند.
\\
\subsection{ د.}
تابع تبدیل سیستم حلقه بسته به صورت زیر خواهد بود:
\[
G(s)=\frac{s+2}{s(s+3)(s+6)}
\]
\[
L(s)= \frac{G(s)}{1+KG(s)} =
\]

در اینجا سیستم اصلا مرتبه ی دو نیست که زتا و 	فرکانس طبیعی تعریف شود.
با این فرض که یک قطب غیر قالب داشته باشیم که از آن صرف نظر کنیم میتوان مقادیری برای زتا و فرمانس طبیعی متصور بود.
\subsection{ه.}

\begin{figure}[h]
	\centering
	\includegraphics[scale=0.35]{6-3.jpg}
\end{figure}

\begin{figure}[h]
	\centering
	\includegraphics[scale=0.35]{6-1.jpg}
\end{figure}

\begin{figure}[h]
	\centering
	\includegraphics[scale=0.35]{6-2.jpg}
\end{figure}

\begin{figure}[h]
	\centering
	\includegraphics[scale=0.35]{6-4.jpg}
\end{figure}




\begin{lstlisting}[caption={Stable region plot (k1 - k2)}]
% Stable region (k1-k2)
clear; close all; clc;

k1 = linspace(-3, 2.99, 500);
k2_upper = -k1.^2 + k1 + 3;
k2_upper = min(k2_upper, 6);      % محدودیت بالایی K2 <= 6
valid_idx = k2_upper > 0;
k1_valid = k1(valid_idx);
k2_upper_valid = k2_upper(valid_idx);

K1g = linspace(min(k1_valid), max(k1_valid), 500);
K2g = linspace(0, max(k2_upper_valid), 500);
[K1, K2] = meshgrid(K1g, K2g);

region = (K2 > 0) & (K2 < (-K1.^2 + K1 + 3)) & (K2 < 6) & (K1 < 3);

figure('Name','Stable Region k1-k2','NumberTitle','off');
contourf(K1, K2, region, [0.5 1], 'LineColor','none');
colormap([0.95 0.98 1; 0.6 0.8 1]);
hold on;
plot(k1, k2_upper, '--', 'Color', [1 0.3 0.3], 'LineWidth', 2);      % مرز بسته
plot([min(k1) max(k1)], [0 0], '-', 'Color', [0.9 0.9 0.9], 'LineWidth', 1);
plot(k1, 6*ones(size(k1)), '--', 'Color', [0.2 1 0.4], 'LineWidth', 1.5);
xlabel('k_1'); ylabel('k_2');
title('Stable Region under Constraints');
xlim([min(k1) max(k1)]);
ylim([min(K2g) max(K2g)]);
grid on; box on;
legend({'Stable Region','Closed-loop Boundary','Lower Limit','Open-loop Limit'}, 'Location','northwest');
hold off;
\end{lstlisting}

\section{چند مثال ریشه‌لوکوس (Root Locus) — مثال‌های کوتاه}
توضیح: چند سیستم مختلف را برای بررسی روت-لوکوس رسم می‌کنیم؛ برای مقایسه، روت‌لوکوس \(G\) و \(-G\) را با هم نشان می‌دهیم.

\begin{lstlisting}[caption={Root locus examples (multiple systems)}]
% RL examples: several quick systems
clear; close all; clc;

% Example 1
num1 = [1 3];
den1 = [1 13 54 82 60 0];
sys1 = tf(num1, den1);
figure; rlocus(sys1); hold on; rlocus(-sys1,'--'); title('RL: sys1 (num=[1 3])'); grid on;
set(findall(gcf,'type','line'),'linewidth',2); hold off;

% Example 2
num2 = [5 1];
den2 = [1 6 5 0 0];
sys2 = tf(num2, den2);
figure; rlocus(sys2); hold on; rlocus(-sys2,'--'); title('RL: sys2 (num=[5 1])'); grid on;
set(findall(gcf,'type','line'),'linewidth',2); hold off;

% Example 3
num3 = [1 -3 2];
den3 = [1 4 0];
sys3 = tf(num3, den3);
figure; rlocus(sys3); hold on; rlocus(-sys3,'--'); title('RL: sys3 (num=[1 -3 2])'); grid on;
set(findall(gcf,'type','line'),'linewidth',2); hold off;
\end{lstlisting}

\section{تحلیل و طراحی برای \(G(s)=1/[s(s+3)(s+6)]\)}
این بخش طولانی‌تر است و شامل موارد زیر می‌شود:
\begin{itemize}
  \item رسم روت-لوکوس سیستم اصلی
  \item یافتن نقاط breakaway و مقدار \(K\) متناظر
  \item تعیین تقاطع با محور موهومی و مقدار بحرانی \(K\)
  \item طراحی PD برای قرار دادن قطب‌های بسته در نقطهٔ مطلوب \(s_d\)
  \item محاسبهٔ پاسخ پله برای چند مقدار \(K\) و برای PD طراحی‌شده
  \item محاسبهٔ حاشیه‌های فرکانسی (GM, PM)
  \item یک فاز بهینه‌سازی ساده برای fine-tuning
\end{itemize}

\begin{lstlisting}[caption={Detailed PD design and analysis for G(s)=1/[s(s+3)(s+6)]}]
% Detailed analysis & PD design for G(s)=1/[s(s+3)(s+6)]
clear; close all; clc;

% System
numG = 1;
denG = [1 9 18 0];   % s*(s+3)*(s+6)
G = tf(numG, denG);

% Desired specs (example: Mp=20%, Ts ~ 2s)
zeta = 0.4559498108;
wn = 4.3864477027;
sd = -zeta*wn + 1i*wn*sqrt(1-zeta^2);  % desired pole
sd_conj = conj(sd);

fprintf('Desired pole sd = %.6f %+.6fi\n', real(sd), imag(sd));

% Check if sd can be placed by only K (no PD)
K_noPD_complex = - sd * (sd + 3) * (sd + 6);  % -1/G(sd)
K_noPD_mag = abs(K_noPD_complex);
fprintf('Without PD: K_required (complex) = %.8f %+.8fi  |K| = %.8f\n', real(K_noPD_complex), imag(K_noPD_complex), K_noPD_mag);
if abs(imag(K_noPD_complex)) > 1e-10
    fprintf('=> K_required is complex -> no real K can place sd on RL (need PD)\n');
else
    fprintf('=> K_required is real -> possible with gain only\n');
end

% Breakaway candidates (roots of dK/ds) (analytical for cubic factors)
roots_break = roots([-3 -18 -18]); % from (s+3)(s+6) etc - original derivation in notes
Kvals_break = -roots_break .* (roots_break + 3) .* (roots_break + 6);
fprintf('Breakaway candidates and K(s):\n');
for i=1:length(roots_break)
    fprintf(' s = %.8f %+.8fi   K(s) = %.8f %+.8fi\n', real(roots_break(i)), imag(roots_break(i)), real(Kvals_break(i)), imag(Kvals_break(i)));
end

% Imag-axis crossing (example numeric result)
w_crit = sqrt(18);
K_crit = 162;
fprintf('Imag-axis crossing at w = %.6f rad/s with K = %.6f\n', w_crit, K_crit);

% Plot RL and some markers
figure('Name','Root Locus and markers','NumberTitle','off');
rlocus(G); hold on; grid on; title('Root Locus of G(s)=1/[s(s+3)(s+6)]');
p = pole(G);
z = zero(G);
plot(real(p), imag(p), 'x','MarkerSize',10,'LineWidth',2); % poles
% mark breakaway candidates if real
r = roots([-3 -18 -18]); r = sort(real(r));
Kvals = -r .* (r+3) .* (r+6);
idx_pos = find(real(Kvals) > 0);
plot(real(r(idx_pos)), zeros(size(idx_pos)), 'ko', 'MarkerFaceColor','y');

% Candidate K values -> closed-loop step responses
Ks = [1, 10.39230485, 50, 162*0.9, 200];
figure('Name','Closed-loop Step Responses','NumberTitle','off');
for k = 1:length(Ks)
    sys_cl = feedback(Ks(k)*tf(numG,denG),1);
    subplot(length(Ks),1,k);
    step(sys_cl,0:0.01:5);
    title(sprintf('Closed-loop Step Response, K=%.4g', Ks(k)));
    grid on;
end

% Design PD to place sd: numeric results from angle/magnitude method
z_pd = 4.5292106756;
K_pd = 21.2409234486;
C_pd = tf([1 z_pd], 1);
L_pd = K_pd * C_pd * G;
T_pd = feedback(L_pd, 1);

fprintf('\nClosed-loop poles with PD (K_pd = %.6f):\n', K_pd);
p_cl_pd = pole(T_pd);
for ii=1:length(p_cl_pd)
    fprintf(' (%.6f %+.6fi)\n', real(p_cl_pd(ii)), imag(p_cl_pd(ii)));
end

% Step response with PD
figure('Name','Step responses: before and after PD','NumberTitle','off');
subplot(2,1,1); step(feedback(10*G,1), 0:0.01:5); title('Step response (no PD), K=10'); grid on;
subplot(2,1,2); step(T_pd, 0:0.01:8); title(sprintf('Step response with PD: K=%.4f, z=%.4f', K_pd, z_pd)); grid on;

% Margins
[GM_pd, PM_pd, Wcg_pd, Wcp_pd] = margin(L_pd);
fprintf('Frequency margins with PD: GM=%.3f dB, PM=%.3f deg\n', 20*log10(GM_pd+eps), PM_pd);

% Fine-tuning via fminsearch (objective: Mp and Ts)
Mp_target_pct = 20;  % percent
Ts_target = 2;       % seconds
w1 = 1; w2 = 0.5;
obj = @(x) local_obj(x, G, Mp_target_pct, Ts_target, w1, w2, K_pd);
x0 = [z_pd, 1];
opts = optimset('Display','off','TolX',1e-3,'TolFun',1e-3);
[xopt,fval] = fminsearch(obj, x0, opts);
z_opt = xopt(1);
K_opt = xopt(2)*K_pd;
fprintf('\nFine-tuning result: z_opt=%.6f, K_opt=%.6f\n', z_opt, K_opt);
T_opt = feedback(K_opt*tf([1 z_opt],1)*G,1);
info_opt = stepinfo(T_opt);
fprintf(' After tuning: Mp=%.3f%%, Ts=%.3fs\n', info_opt.Overshoot, info_opt.SettlingTime);
\end{lstlisting}

\section{توابع کمکی مورد استفاده در اسکریپت بالا}
توضیح: تابع local\_obj برای بهینه‌سازی جزئی (fine-tuning) استفاده شده — این تابع را در همان فایل یا به‌عنوان تابع محلی انتهای فایل قرار بده.

\begin{lstlisting}[caption={local\_obj (helper for fine-tuning)}]
function J = local_obj(x, G, Mp_target_pct, Ts_target, w1, w2, Kpd_nom)
    z_try = x(1);
    Kmult = x(2);
    % محدودیت‌های ساده: زیرو بهره باید مثبت باشند
    if Kmult <= 0 || z_try <= 0
        J = 1e6 + 1e3*abs(Kmult) + 1e3*abs(z_try);
        return;
    end
    C_try = tf([1 z_try],1);
    T_try = feedback(Kmult * Kpd_nom * C_try * G, 1);
    info = stepinfo(T_try);
    if isempty(info.Overshoot) || isempty(info.SettlingTime) || isnan(info.SettlingTime) || info.SettlingTime<=0
        J = 1e6 + 1e3*abs(info.SettlingTime);
    else
        Mp_try = info.Overshoot;
        Ts_try = info.SettlingTime;
        J = w1*(Mp_try - Mp_target_pct)^2 + w2*(Ts_try - Ts_target)^2;
    end
end
\end{lstlisting}


\end{document}

