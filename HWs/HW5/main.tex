\documentclass{article}

\usepackage{graphicx}
\usepackage{fancyhdr}
\usepackage[sorting=none]{biblatex}
\usepackage[margin=1in]{geometry}
\usepackage{listings}
\usepackage{courier}
\usepackage[hidelinks]{hyperref}
\usepackage{subfigure}
\hypersetup{
    colorlinks=true,
    linkcolor=teal,
    filecolor=magenta,      
    urlcolor=teal,
    citecolor = teal
    }
\usepackage{xcolor}
\usepackage{xepersian}
\setlength\headheight{28pt} 
\addbibresource{bibliography.bib}
\settextfont[Path={./font/}, Scale=1.3]{IRLotus}
\setlatintextfont[Scale=1]{Times New Roman}
\usepackage{float} 
\renewcommand{\baselinestretch}{1.5}
\pagestyle{fancy}
\fancyhf{}
\rhead{\includegraphics[width=1cm]{img/Logo.png} پاسخ تمرین 5 }
\lhead{\thepage}
\renewcommand{\headrulewidth}{1pt}
\renewcommand{\footrulewidth}{1pt}
\AtBeginDocument{
	\def\chapterautorefname{فصل}%
	\def\sectionautorefname{پاسخ سوال}%
	\def\subsectionautorefname{بخش}%
	\def\subsubsectionautorefname{بخش}%
	\def\equationautorefname{رابطهٔ}%
    \def\lstlistingautorefname{برنامۀ}%
}
\renewcommand{\lstlistingname}{Code}

\definecolor{codegreen}{rgb}{0,0.6,0}
\usepackage{amsmath}
\definecolor{codegray}{rgb}{0.5,0.5,0.5}
\definecolor{codepurple}{rgb}{0.58,0,0.82}
\definecolor{backcolour}{rgb}{0.95,0.95,0.92}

\lstdefinestyle{mystyle}{
	backgroundcolor=\color{backcolour},   
	commentstyle=\color{codegreen},
	keywordstyle=\color{magenta},
	numberstyle=\tiny\color{codegray},
	stringstyle=\color{codepurple},
	basicstyle=\ttfamily\footnotesize,
	breakatwhitespace=false,         
	breaklines=true,                 
	captionpos=b,                    
	keepspaces=true,                 
	numbers=left,                    
	numbersep=5pt,                  
	showspaces=false,                
	showstringspaces=false,
	showtabs=false,                  
	tabsize=2
}

\lstset{
    language=Matlab,
    basicstyle=\footnotesize\ttfamily,
    keywordstyle=\color{blue},
    commentstyle=\color{green!50!black},
    stringstyle=\color{red},
    numbers=left,
    numberstyle=\tiny,
    stepnumber=1,
    numbersep=5pt,
    backgroundcolor=\color{gray!10},
    frame=single,
    breaklines=true,
    tabsize=4,
    showstringspaces=false
}


\lstset{style=mystyle}

\begin{document}

\begin{titlepage}
\begin{center}
\defpersianfont\nast[Path={./font/}, Scale=2]{IranNastaliq}
\centerline{{\includegraphics[width=5cm]{img/Logo.png}}}
\centerline{\textcolor[rgb]{0,0,0.5}{\nast \large  دانشگاه صنعتی خواجه نصیرالدین طوسی}}
\centerline{\textcolor[rgb]{0,0,0.5}{\nast \bfseries دانشکدۀ مهندسی برق - گروه مهندسی کنترل}}

\vfill
        
\Huge
\textbf{کنترل خطی}\\
\textbf{پاسخ تمرین 5}\\
        
\vfill
        
\begin{table}[ht]
    \centering
    \huge
    \begin{tabular}{|c|c|}
    \hline
    نام و نام خانوادگی & محمد امین حری فراهانی\\
    \hline
    شمارۀ دانشجویی & 40216674\\
    \hline
    تاریخ & آذر 1404\\
    \hline
    \end{tabular}
\end{table}
\end{center}
\end{titlepage}


\tableofcontents \clearpage
\listoffigures \clearpage
\clearpage
 \clearpage
\newpage



\section{سوال 1 }\label{Section1}

\begin{equation}
G(s)=\frac{(s^{2}-2s+2)(5s+1)}{s^{2}+2s+2}
\end{equation}


\subsection{الف ) کشیدن نمودار بودی}
% --- اگر تصویر PDF داری ---
\begin{figure}[H]
  \centering
  \includegraphics[width=0.88\textwidth]{img/111.jpg}
  \caption{رسم دستی}
  \label{fig:block-diagram}
\end{figure}

\subsection{رسم با استفاده از متلب}

\begin{figure}[H]
  \centering
  \includegraphics[width=0.88\textwidth]{img/112.png}
  \caption{Magnitude}
  \label{fig:block-diagram}
\end{figure}
\begin{figure}[H]
  \centering
  \includegraphics[width=0.88\textwidth]{img/113.png}
  \caption{Phase}
  \label{fig:block-diagram}
\end{figure}
مقایسه با رسم دستی نشان داد که شکل‌ها تقریباً مشابه هستند .

\subsection{ج ) نایکوئیست و بررسی پایداری}
\vspace*{1cm} 
کد متلب :
\begin{lstlisting}
% تعریف صورت و مخرج تابع تبدیل
num = conv([1 -2 2], [5 1]); % صورت: (s^2 - 2s + 2)(5s + 1)
den = [1 2 2];               % مخرج: s^2 + 2s + 2

% ساخت تابع تبدیل
G = tf(num, den);

% رسم نمودار نایکوییست
figure;
nyquist(G);
grid on;
title('Nyquist Plot of G(s)');
\end{lstlisting}
\vspace*{2cm} 
\begin{figure}[H]
  \centering
  \includegraphics[width=0.88\textwidth]{img/114.png}
  \caption{نمودار نایکوئیست سیستم}
  \label{fig:block-diagram}
\end{figure}

\section*{تحلیل نایکوییست و پایداری سیستم}

تابع تبدیل حلقه باز:
\[
G(s)=\frac{(s^{2}-2s+2)(5s+1)}{s^{2}+2s+2}
\]

\subsection*{قطب‌ها و صفرها}
صورت و مخرج را جدا می‌کنیم:
\[
\text{num}(s)=(s^2-2s+2)(5s+1),\quad \text{den}(s)=s^2+2s+2
\]

ریشه‌های مخرج (قطب‌های باز) با حل \(s^2+2s+2=0\) به‌دست می‌آیند:
\[
s=-1\pm j1 \quad (\text{هر دو در نیم‌صفحهٔ چپ})
\]

ریشه‌های صورت:
\[
s^2-2s+2=0 \Rightarrow s=1\pm j1 \quad (\text{دو صفر در نیم‌صفحهٔ راست}),
\]
و همچنین از \(5s+1=0\) داریم:
\[
s=-0.2 \quad (\text{صفر در نیم‌صفحهٔ چپ})
\]

بنابراین تابع باز \(G(s)\) دارای \(P=0\) قطب در نیم‌صفحهٔ راست و ۲ صفر در RHP است (سیستم غیرمینیمم‌فاز).

\subsection*{تحلیل نایکوییست}
برای تعیین نقاط تقاطع نمودار نایکوییست با محور حقیقی، بخش موهومی \(Im\{G(j\omega)\}=0\) قرار داده می‌شود:
\[
5\omega^5 - 40\omega^3 + 20\omega = 0
\]
حل معادله فوق نتایج زیر را می‌دهد:
\[
\omega = 0, \quad \omega = \pm 0.7, \quad \omega = \pm 2.73
\]

تابع تبدیل سیستم قطب ناپایدار ندارد، بنابراین:
\[
P = 0
\]

طبق معیار نایکوییست:
\[
Z = N + P
\]
که در آن \(Z\) تعداد صفرهای حلقه بسته در RHP (داخل منحنی نایکوییست) و \(N\) تعداد دورانی‌های نمودار حول نقطه \(-1/K\) است.

\subsection*{نتیجه‌گیری برای بهره‌های مختلف \(K\)}
با توجه به نمودار نایکوییست:
\begin{itemize}
    \item به ازای \(K < 0.36\)، نمودار نقطه \(-1/K\) را دور نمی‌زند، بنابراین سیستم حلقه بسته پایدار است.
    \item به ازای \(K > 0.36\)، نمودار حول نقطه \(-1/K\) دوران می‌کند و سیستم وارد ناپایداری می‌شود.
\end{itemize}

\section{سوال 2 }\label{Section1}
\textbf{خلاصهٔ مسئله:}\\
یک نمودار Bode داده‌شده است که سه نقطهٔ مشخص روی آن عبارت‌اند از:
\begin{itemize}
  \item $\omega_1 \approx 0.117\ \mathrm{rad/s},\ M(\omega_1)=61.4\ \mathrm{dB}$
  \item $\omega_2 \approx 3.93\ \mathrm{rad/s},\ M(\omega_2)=1.14\ \mathrm{dB}$
  \item $\omega_3 \approx 258\ \mathrm{rad/s},\ M(\omega_3)=-70.4\ \mathrm{dB}$
\end{itemize}
و فاز در نقاط:
\[
\angle G(0.108)\approx -2.79^\circ,\quad \angle G(4.43)\approx -89.4^\circ,\quad \angle G(255)\approx -178^\circ.
\]


\section*{۱. نقاط استفاده‌شده و تبدیل دسی‌بل به قدر مطلق}
برای تبدیل مقدار دسی‌بل به قدر مطلقِ فرکانسی از رابطهٔ
\[
|G(j\omega)| = 10^{M(\omega)/20}
\]
استفاده شده است:
\begin{align*}
|G(j0.117)| &\approx 10^{61.4/20}\approx 1174.9,\\
|G(j3.93)|  &\approx 10^{1.14/20}\approx 1.1403,\\
|G(j258)|   &\approx 10^{-70.4/20}\approx 3.0199\times 10^{-4}.
\end{align*}

\section*{آ) وجود صفر و مینیمم‌فاز بودن}
\begin{itemize}
  \item رفتار فاز: فاز از تقریباً $0^\circ$ در فرکانس پایین تا حدود $-180^\circ$ در فرکانس بالا به‌صورت یکنواخت نزول می‌کند؛ این الگو شبیه رفتار دو قطب منفرد در \textbf{نیم‌صفحهٔ چپ} است.
  \item اگر صفرِ RHP (نیم‌صفحهٔ راست) وجود داشت، معمولاً در فاز افزایشِ مثبت یا تغییر نامعمول مشاهده می‌شد. چنین چیزی در نمودار نیست.
  \item نتیجه: \textbf{هیچ نشانهٔ آشکاری از صفر RHP وجود ندارد. سیستم به‌صورت محتاطانه مینیمم‌فاز در نظر گرفته می‌شود.}
\end{itemize}

\section*{ب) خطای حالت ماندگار (با فرض حلقهٔ بسته unity feedback)}
ابتدا نوع سیستم (System Type) را تعیین می‌کنیم:
\begin{itemize}
  \item فاز در فرکانس پایین تقریباً $0^\circ$ است؛ بنابراین \textbf{هیچ انتگراتور (pole at origin)} دیده نمی‌شود $\Rightarrow$ سیستم از نوع \textbf{Type-0}.
\end{itemize}

برای سیستم Type-0 در حلقهٔ بسته unity:
\begin{itemize}
  \item ثبات موقعیت: $K_p=\lim_{s\to 0}G(s)\approx |G(j0.117)|\approx 1174.9$ (برآورد از مقدار فرکانس پایین)
  \item خطای پله:
  \[
  e_{ss,\text{step}}=\frac{1}{1+K_p}\approx \frac{1}{1+1174.9}\approx 0.00085\ (\approx 0.085\%).
  \]
  \item خطای شیب (ramp) و سهموی (parabolic) برای Type-0 \textbf{نامتناهی} است:
  \[
  e_{ss,\text{ramp}}=\infty,\qquad e_{ss,\text{parabolic}}=\infty.
  \]
\end{itemize}

\section*{ج) تقریب تابع تبدیل \(G(s)\)}
با توجه به شیب تقریباً \(-40\ \mathrm{dB/dec}\) و فازِ نهایی $\approx -180^\circ$، فرض می‌کنیم مدل ساده دو قطب LHP و بدون صفر مؤثر باشد:
\[
G(s)\approx \frac{K}{(s+p_1)(s+p_2)}.
\]

از نقطهٔ فرکانس بالا تقریب می‌زنیم که برای $\omega\gg p_i$ داریم $|G(j\omega)|\approx \dfrac{K}{\omega^2}$. بنابراین با استفاده از نقطه $\omega=258$:
\[
K \approx |G(j258)|\cdot 258^2 \approx 3.02\times10^{-4}\cdot 258^2 \approx 20.1.
\]
سپس با برازش عددی ساده روی سه نقطهٔ داده‌شده، یک تقریب مناسب برای بازتولید مگنی‌تود به‌دست آمد:
\[
\boxed{G(s)\approx \dfrac{19.3}{(s+0.0518)^2}.}
\]


\section{سوال 3 }\label{Section1}

\section*{۱. قطب‌ها و صفرها}
مخرج تابع برابر است با $s(0.1s+1)(0.5s+1)$، لذا قطب‌ها:
\[
s=0,\qquad s=-10,\qquad s=-2.
\]
تابع حلقهٔ باز صفر ندارد.

\section*{۲. معادلهٔ مشخصه (حلقهٔ بسته)}
معادلهٔ مشخصه:
\[
1+L(s)=0 \quad\Longrightarrow\quad s(0.1s+1)(0.5s+1)+100k=0.
\]
با گسترش:
\[
0.05 s^3 + 0.6 s^2 + s + 100k = 0.
\]
ضرب در $20$:
\[
\boxed{s^3 + 12 s^2 + 20 s + 2000 k = 0.} \tag{*}
\]

\section*{۳. جدول راث}
برای چندجمله‌ای \((*)\) جدول راث ستون اول به صورت زیر است:
\[
\begin{array}{c|cc}
s^3 & 1 & 20\\[2pt]
s^2 & 12 & 2000k\\[2pt]
s^1 & b_1 & 0\\[2pt]
s^0 & 2000k & -
\end{array}
\qquad\text{که}\qquad
b_1=\frac{12\cdot 20 - 1\cdot 2000k}{12}=20-\frac{500}{3}k.
\]

شرط پایداری (همهٔ عناصر ستون اول مثبت) حکم می‌کند:
\[
2000k>0 \Rightarrow k>0,\qquad b_1>0 \Rightarrow k<0.12.
\]

\paragraph{بنابراین:}
\[
\boxed{0 < k < 0.12 \quad \text{(حلقهٔ بسته پایدار).}}
\]
در $k=0.12$ حالت مرزی (دو ریشه موهومی)، برای $k>0.12$ ناپایدار.

\section*{۴. نقطهٔ عبور از محور موهومی (ناکوییست / فاز)}
برای $s=j\omega$ زاویهٔ $L(j\omega)$ برابر است با:
\[
\angle L(j\omega)=-90^\circ - \arctan(0.1\omega) - \arctan(0.5\omega).
\]
شرط فازی برای برخورد با $-1$:
\[
-90^\circ - \arctan(0.1\omega) - \arctan(0.5\omega) = -180^\circ
\]
یعنی
\[
\arctan(0.1\omega)+\arctan(0.5\omega)=90^\circ.
\]
با استفاده از فرمول جمع تانژانت می‌یابیم:
\[
1-0.05\omega^2=0 \quad\Rightarrow\quad \omega=\sqrt{20}\approx 4.4721.
\]
اندازهٔ $L(j\omega)$ در این فرکانس:
\[
|L(j\omega)|=\frac{100k}{\omega\sqrt{1+(0.1\omega)^2}\sqrt{1+(0.5\omega)^2}}.
\]
شرط $|L|=1$ برای برخورد به $-1$ نتیجه می‌دهد:
\[
k_{\text{cr}}\approx 0.12.
\]

\section*{۵. نکات روت‌لوکوس}
\begin{itemize}
    \item قطب‌ها: $0,-2,-10$. هیچ صفر باز ندارد.
    \item مرکز آسمپتوت‌ها: $\sigma_a=(-12)/3=-4$.
    \item زوایا: $60^\circ,180^\circ,300^\circ$.
    \item روت‌لوکوس روی محور حقیقی در \((-\infty,-10)\) و \((-2,0)\).
    \item نقطهٔ breakaway در بازهٔ $(-2,0)$ تقریباً در $s\approx -0.94495$، مقدار متناظر $k\approx 0.0045138$.
    \item آستانهٔ پایداری: $k=0.12$ (دو ریشه در $s=\pm j4.472$).
\end{itemize}

\section*{۶. رسم نایکوئیست}
\begin{figure}[H]
  \centering
  \includegraphics[width=0.88\textwidth]{img/115.jpg}
  \caption{نمودار نایکوئیست }
  \label{fig:block-diagram}
\end{figure}


\section*{۷. نتیجهٔ }
\[
\boxed{0<k<0.12 \quad \text{حلقهٔ بسته پایدار است.}}
\]
در $k=0.12$ حالت مرزی با جفت ریشهٔ موهومی در $\pm j4.472$. برای مقدارهای بزرگ‌تر از آن سیستم ناپایدار می‌شود.


\section{سوال 4 }\label{Section1}

\begin{figure}[H]
  \centering
  \includegraphics[width=0.88\textwidth]{img/222.jpg}
  \caption{نمودار نایکوئیست }
  \label{fig:block-diagram}
\end{figure}
\begin{figure}[H]
  \centering
  \includegraphics[width=0.88\textwidth]{img/223.jpg}
  \caption{نمودار نایکوئیست }
  \label{fig:block-diagram}
\end{figure}
\begin{figure}[H]
  \centering
  \includegraphics[width=0.88\textwidth]{img/224.jpg}
  \caption{نمودار نایکوئیست }
  \label{fig:block-diagram}
\end{figure}


\section*{۱. استخراج معادلهٔ مشخصه}
اگر $P(s)$ انتقالِ جسم (plant) باشد:
\[
P(s)=\frac{10}{1+0.1s}\cdot\frac{1}{0.01 s^2}=\frac{1000}{s^2(1+0.1s)}.
\]
با نوشتن روابط حلقه (تفاضل‌ها و فیدبک داخلی) معادلهٔ مشخصه به صورت
\[
1 + (K + K_t s)\,P(s)=0
\]
به‌دست می‌آید. با ضرب در مخرج این عبارت تبدیل می‌شود به:
\[
s^2(1+0.1s) + 1000(K + K_t s) = 0,
\]
یا به شکل چندجمله‌ای درجهٔ سه:
\[
0.1 s^3 + s^2 + 1000 K_t s + 1000 K = 0.
\]
ضرب در $10$ برای حذف اعشار:
\[
\boxed{s^3 + 10 s^2 + 10000 K_t s + 10000 K = 0.}
\]

\section*{۲. جدول راث}
اضلاع: $a_3=1,\; a_2=10,\; a_1=10000K_t,\; a_0=10000K$. جدول راث ستون اول:
\[
\begin{array}{c|cc}
s^3 & 1 & 10000K_t\\[4pt]
s^2 & 10 & 10000K\\[4pt]
s^1 & b_1 & 0\\[4pt]
s^0 & 10000K & -
\end{array}
\qquad
b_1 = \frac{10\cdot 10000K_t - 1\cdot 10000K}{10} = 10000K_t - 1000K.
\]

شرط پایداری: تمامی اعضای ستون اول مثبت، بنابراین:
\[
10000K>0 \Rightarrow K>0,
\]
و
\[
10000K_t - 1000K>0 \Rightarrow 10K_t - K >0.
\]
نتیجهٔ نهایی:
\[
\boxed{K>0 \quad\text{و}\quad K<10K_t.}
\]

\section*{۳. پاسخ‌های موردی}
\begin{itemize}
  \item[(الف)] $K_t=0$: آنگاه $K<0$ از یک سو و $K>0$ از سوی دیگر مورد نیاز است — تناقض. بنابراین \textbf{هیچ $K>0$‌ای سیستم را پایدار نمی‌کند}. ($K=0$ حالت مرزی است.)
  \item[(ب)] $K_t=0.01$: شرط $\;0<K<10K_t\;$ می‌دهد $0<K<0.1$.
  \item[(ج)] $K_t=0.1$: شرط $\;0<K<10K_t\;$ می‌دهد $0<K<1$.
  \item[(د)] اگر $K=10$ آنگاه شرط $K<10K_t$ می‌شود $10<10K_t$ یا $K_t>1$. پس برای $K=10$ باید $K_t>1$ باشد تا سیستم پایدار شود.
\end{itemize}





\section{ضمیمه}
\href{https://github.com/aminism00/LinearControl2025/tree/main/HWs/HW5}{ ادرس گیت هاب  (\lr{GitHub})}.

\end{document}
